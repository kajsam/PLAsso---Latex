\documentclass[12pt]{article}

% Packages
\usepackage{amssymb}
\usepackage{amsmath} % align
\usepackage{color}
\usepackage{graphicx}
\usepackage{subfig}
\usepackage{multirow}
\usepackage{centernot}
\usepackage{setspace}


\DeclareMathOperator*{\argmin}{arg\,min}


\begin{document}

\newcommand{\cpos}{\text{\scriptsize{{\it POS}}}}
\newcommand{\cneg}{\text{\scriptsize{{\it NEG}}}}
\newcommand{\mb}{\mathbf}

\section{Simulated matrices} \label{sec:Sim} %%%%%%%%%%%%%%%%%%%%%%%%%%%%%%%%%%%%

To simulate a two-class $n \times d$  single-cell RNA matrix, where $n$ is the number of cells and $d$ is the number of genes, we define an $n \times d$ binary structure matrix $\mathbf{S}$ with entries $s_{ij}$ and an $n \times d$ probability matrix $P$. 

The entries of the probability matrix are
\begin{align}
  p_{ij} = Pr(X_{ij} = s_{ij}),
\end{align}
where $Pr$ is the probability function. 
The simulated matrix $\mathbf{X}$ is binary, with entries
\begin{align}
  x_{ij} &= 
  \begin{cases}
   1 &  \text{ if gene $j$ is expressed in cell $i$} \\
   0 & \text{ otherwise.}
  \end{cases}
\end{align}
Let each entry be the observation of a random vector $X_{ij} \sim Bernoulli(\pi_{ij})$, and let 
\begin{align}
  \pi_{ij} = (1 -  p_{ij}) + (2  p_{ij} - 1) s_{ij}.
\end{align}

\subsection{The structure matrix} \label{sec:Struct} %%%%%%%%%%%%%%%%%%%%%%%%%%%%%%%%%%%%%%%%%%

The structure matrix $\mathbf{S}$ represents a simplified/idealised underlying structure of the cell-gene interaction according to the following assumptions:\\
(1) Each cell belongs to one of the two classes, class 1 or class 2.\\
(2) Each gene is expressed in either \\
\indent (a) all cells in both classes \\
\indent (b) all cells in class 1,  none of the cells in class 2 \\
 \indent (c) all cells in class 2, none of the cells in class 1 \\
 \indent (d) none of the cells.

This can be expressed as follows:
Let $\mathbb{Z}_n$ and $\mathbb{Z}_d$ be the sets of integers from $1$ to $n$ and $1$ to $d$, respectively. 
Let $\mathbb{C}_1 \subset \mathbb{Z}_n$ be the set of cells in class 1.
Then $\mathbb{C}_2 = \mathbb{C}_{12} - \mathbb{C}_1$, where $\mathbb{C}_{12} = \mathbb{Z}_n$ (assumption 1) and $-$ is the set difference operator. 
Let $\mathbb{G}_{12} \subset \mathbb{Z}_d$ be the genes expressed for both classes, let $\mathbb{G}_{1} \subset \mathbb{Z}_d$ be the genes expressed for class 1, and let $\mathbb{G}_{2} \subset \mathbb{Z}_d$ be the genes expressed for class 2. 
Then $\mathbb{G}_{12} \cap \mathbb{G}_{1} = \emptyset$,  $\mathbb{G}_{12} \cap \mathbb{G}_{2} = \emptyset$,  $\mathbb{G}_{1} \cap \mathbb{G}_{2} = \emptyset$ (assumption 2).

Let $\mathbb{A}$ be a set of positive integers, and let $\mathbf{1}_{\mathbb{A}}$ be a column vector where each entry $i$
\begin{align}
  \begin{cases}
   1 &  \text{ if } i \in \mathbb{A} \\
   0 & \text{ otherwise.}
  \end{cases}
\end{align}
Then we can write the structure matrix $\mathbf{S}$ as 
\begin{align}
  \mathbf{S} = \mathbf{S}_{12}+ \mathbf{S}_1 + \mathbf{S}_2 = \mathbf{1}_{\mathbb{C}_{12}}\mathbf{1}^{\prime}_{\mathbb{G}_{12}} +  \mathbf{1}_{\mathbb{C}_{1}}\mathbf{1}^{\prime}_{\mathbb{G}_{1}} +  \mathbf{1}_{\mathbb{C}_{2}}\mathbf{1}^{\prime}_{\mathbb{G}_{2}},
\end{align}
where $\prime$ denotes the transpose, and $+$ is the boolean $or$ operator. 
Equivalently
\begin{align}
  \mathbf{S} = [\mathbf{1}_{\mathbb{C}_{1}} \mathbf{1}_{\mathbb{C}_{2}}] [\mathbf{1}^{\prime}_{\mathbb{G}_{12} \cup {\mathbb{G}_{1}}}; \mathbf{1}^{\prime}_{\mathbb{G}_{12} \cup {\mathbb{G}_{2}}}],
\end{align}
where $[\mathbf{a} \mathbf{b}]$ is the concatenation of columns $\mathbf{a} \mathbf{b}$ into a matrix, and $[\mathbf{a}\prime ; \mathbf{b}\prime]$ is the concatenation of rows into a matrix. 

\subsection{The probability matrix} \label{sec:Prob} %%%%%%%%%%%%%%%%%%%%%%%%%%%%%%%%%%%%%%%%%%

Three simplified cases. 

{\bf No cell effect, no gene effect:} $p_{ij} = p$.
The probability is the same for all cells and genes. 

{\bf Class cell effect, no gene effect:} $ p_{class 1},\,  p_{class 2}$
\begin{align}
  p_{ij} =
  \begin{cases}
    p_{class 1} \text{ if } i \in \mathbb{C}_{1} \\
    p_{class 2} \text{ if } i \in \mathbb{C}_{2}
  \end{cases}
\end{align}

{\bf Gene effect, no interaction:} $ p_{gene 12}, \, p_{gene 1}, \, p_{gene 2}$
\begin{align}
  p_{ij} =
  \begin{cases}
    p_{gene 12} \text{ if } j \in \mathbb{G}_{12} \\
    p_{gene 1} \text{ if } i \in \mathbb{G}_{1} \\
    p_{gene 2} \text{ if } i \in \mathbb{G}_{2} \\
    p_{gene 0} \text{ otherwise} 
  \end{cases}
\end{align}

{\bf Cell and gene effect, multiplicative:} 
\begin{align}
  p_{ij} =  p_{cell A} \cdot p_{gene B} \text{ if } i  \in \mathbb{C}_{A}, j \in \mathbb{G}_{B}
\end{align}

{\bf Cell and gene effect:} 
\begin{align}
  p_{ij} =  g(p_{cell A}, p_{gene B}) \text{ if } i  \in \mathbb{C}_{A}, j \in \mathbb{G}_{B},
\end{align}
where $g(p_{cell},p_{gene})$ is some function. 


\subsection{Demonstration} \label{sec:Demo} %%%%%%%%%%%%%%%%%%%%%%%%%%%%%%%%%%%%%%%%%%

\begin{figure}
  \includegraphics[width = \textwidth]{S.jpg}
  \caption{The structure matrix}
\end{figure}

\begin{figure}
  \subfloat{ 
  \includegraphics[width = 0.5 \textwidth]{Pi1.jpg}}
   \subfloat{ 
  \includegraphics[width = 0.5 \textwidth]{X1.jpg}}
  \caption{No cell effect, no gene effect}
\end{figure}

\begin{figure}
  \subfloat{ 
  \includegraphics[width = 0.5 \textwidth]{PiCell.jpg}}
   \subfloat{ 
  \includegraphics[width = 0.5 \textwidth]{XCell.jpg}}
  \caption{Class cell effect, no gene effect}
\end{figure}

\begin{figure}
  \subfloat{ 
  \includegraphics[width = 0.5 \textwidth]{PiGene.jpg}}
   \subfloat{ 
  \includegraphics[width = 0.5 \textwidth]{XGene.jpg}}
  \caption{Gene effect, no interaction}
\end{figure}

\begin{figure}
  \subfloat{ 
  \includegraphics[width = 0.5 \textwidth]{PiCellGene.jpg}}
   \subfloat{ 
  \includegraphics[width = 0.5 \textwidth]{XCellGene.jpg}}
  \caption{ Cell and gene effect, multiplicative}
\end{figure}

See matlab functions cell\_gene\_sim.m and simulation\_parameters.m


\end{document}
